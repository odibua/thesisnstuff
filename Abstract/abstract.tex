% ************************** Thesis Abstract *****************************
% Use `abstract' as an option in the document class to print only the titlepage and the abstract.
\begin{abstract}
The construction and validation of models for systems is essential to engineering. This requires accurate mathematical representations of relevant processes, parameter and/or state estimation in order to inform models, and high quality experiments with which to validate these models. Typically, an individual research project in the context of academia focuses on one or two of these steps. In the work done in this thesis, we perform all three steps of model construction and validation, and apply it to the case of an electrostatic comb-drive actuator in electrolytes. \\

Electrostatic comb-drive actuators in electrolytes have many potential applications. Two applications in particular are found in literature. The most commonly cited is characterizing the mechanical properties of diverse biological structures at small scales. The mechanical properties of cells are critical to their health, and their behavior in response to external and internal stimuli. Electrostatic comb-drive actuators in conducting liquids also have potential in microfluidic systems. In particular, they can be used in microfluidic systems designed for drug delivery, and the mechanical manipulation of cells. Maximizing the utility of these devices for such applications requires robust methods for operating electrostatic actuators in electrolytes, and the creation of accurate models for these devices. \\

Electrostatic actuators are typically operated with DC voltages. In electrolytes, an ionic shield forms on the actuator surface's under the influence of DC voltages, preventing motion. It was discovered that this effect could be overcome by applying AC voltages with sufficiently high frequencies. Lumped circuit models were subsequently developed that accurately captured the general behavior of electrostatic actuators with comb-drive and parallel plate configurations respectively. However, these models had limited quantitative accuracy at low and intermediate frequencies of the AC signal. We develop models of the comb-drive actuator in electrolytes that are accurate over a wide range of frequencies, and that will facilitate accurate control of these actuators in both the context of deforming cells, and of acting as actuators in microfluidic systems. 
\end{abstract}
